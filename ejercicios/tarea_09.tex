\documentclass{article}
\usepackage[utf8]{inputenc}
\usepackage[spanish]{babel}
\usepackage{hyperref}
 
\hypersetup{
    colorlinks=true,
    linkcolor=blue,
    filecolor=magenta,      
    urlcolor=blue,
}

\usepackage{listings}
\usepackage{color}

\definecolor{codegreen}{rgb}{0,0.6,0}
\definecolor{codegray}{rgb}{0.5,0.5,0.5}
\definecolor{codepurple}{rgb}{0.58,0,0.82}
\definecolor{backcolour}{rgb}{0.95,0.95,0.92}
\lstdefinestyle{mystyle}{
    backgroundcolor=\color{backcolour}, commentstyle=\color{codegreen}, keywordstyle=\color{magenta},
    numberstyle=\tiny\color{codegray}, stringstyle=\color{codepurple}, basicstyle=\footnotesize,
    breakatwhitespace=false, breaklines=true, captionpos=b, keepspaces=true, numbers=left,                    
    numbersep=5pt, showspaces=false, showstringspaces=false, showtabs=false,tabsize=2
}
\lstset{style=mystyle}

\title{Tarea 9}

\begin{document}

\maketitle

\section{Ejercicio 1 - Ceros de las funciones de Bessel}

Las funciones de Bessel aparecen como soluci\'on a ecuaciones diferenciales 
en ciencias naturales 
\footnote{\url{https://en.wikipedia.org/wiki/Bessel_function}}
Pueden describir la forma de una membrana de un tambor cuando suena o
el potencial eléctrico dentro de cavidades cilíndricas por mencionar
algunos problemas físicos. 
En este ejercicio utilizaremos una librería que calcula las funciones de Bessel.

\subsection{Importando la librería SciPy}
Esta biblioteca contiene m\'etodos que nos permite calcular 
la funci\'on de Bessel $J_\nu(x)$. 

En la primera celda escriba
\begin{lstlisting}[language=Python]
import numpy as np
import matplotlib.pyplot as plt
import scipy.special as sp
%matplotlib inline
\end{lstlisting}

Cree un array tipo linspace $x$ entre 0 y 20 y 100 puntos.
Genere un segundo array $y=J_\nu(x)$, con $\nu = 1$.
\begin{lstlisting}[language=Python]
x = np.linspace(0,20, 100)
y = sp.jv(1,x)
\end{lstlisting}

Grafique $x$ vs $y$.
\begin{lstlisting}[language=Python]
plt.title("Funcion de Bessel de primer orden")
plt.plot(x,y)
plt.grid(True)
plt.axhline(y=0, color="black")
plt.axvline(x=0, color="black")
\end{lstlisting}

\subsection{(30 puntos) Buscando ceros por barrido}

Implemente un algoritmo para buscar ceros a través de cambio de signo
en la función mientras se hace barrido en $x$ desde cero hasta veinte.
Omita $x=0$.

\begin{itemize}
\item El algoritmo debe barrer el dominio de $x$ en pequeños pasos pequeños.
\item El algoritmo debe decir si encuentra ceros a partir del cambio de
  signo de la función
\item El algoritmo debe encontrar los ceros con un error no mayor a 0.2
\end{itemize}

El c\'odigo no puede ser una copia del código en el vídeo. 
Entienda el método y escriba su propia versión.

\subsection{(30 puntos) Buscando ceros por bisección}
Omita $x=0$.

\begin{itemize}
\item El algoritmo buscará en todos los ceros que encontró en el punto anterior.
\item Si se tiene un cero en $x_0$, el algoritmo  puede empezar a buscar
  dentro de un rango $(x_0 - 0.5 , x_0 + 0.5)$
  alrededor de cada cero que había encontrado.  
\item El algoritmo debe dividir el rango entre dos partes, e identificar
  en qué momento cambia la función de signo para aceptar el rango
  donde se hace el cambio de signo y rechazar el dominio donde no
  cambió la función de signo. Esta parte debe ser repetirse hasta
  encontrar el cero dentro del error propuesto (0.0001)
  $|f(x)| < 0.0001$
\item El algoritmo debe encontrar los ceros que encontró en el punto
  anterior.
\item El algoritmo debe reportar cuántas iteraciones requiere para
  encontrar cada uno de los ceros en el dominio.
\end{itemize}

El c\'odigo no puede ser una copia del código en el vídeo. 
Entienda el método y escriba su propia versión.


\subsection{(40 puntos) Método de Newton-Raphson}

Escribir un algoritmo que busque los ceros de la función de Bessel
$J_1(x)$ entre 0 y 20. Omitir $x=0$.

\begin{itemize}
\item El algoritmo debe utilizar los mismos puntos de partida que en el
  punto anterior, cercanos a los ceros.
\item El algoritmo debe calcular la derivada de la funci\'on.
\item A partir de la derivada, debe buscar el intersecto de la recta tangente
  al punto donde se calculó la derivada.
\item Una vez se tiene el intersecto, se evalúa la función, se repite el
  procedimiento hasta que el error sea inferior a 0.0001.
  $|f(x)| < 0.0001$
\item Indicar cuántas iteraciones se requieren para que este método
  encuentre cada uno de los ceros.
\end{itemize}

\end{document}

\end{document}
