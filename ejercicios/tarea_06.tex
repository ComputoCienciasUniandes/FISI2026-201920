\documentclass{article}
\usepackage[utf8]{inputenc}
\usepackage[spanish]{babel}
\usepackage{hyperref}
 
\hypersetup{
    colorlinks=true,
    linkcolor=blue,
    filecolor=magenta,      
    urlcolor=blue,
}

\usepackage{listings}
\usepackage{color}

\definecolor{codegreen}{rgb}{0,0.6,0}
\definecolor{codegray}{rgb}{0.5,0.5,0.5}
\definecolor{codepurple}{rgb}{0.58,0,0.82}
\definecolor{backcolour}{rgb}{0.95,0.95,0.92}
\lstdefinestyle{mystyle}{
    backgroundcolor=\color{backcolour}, commentstyle=\color{codegreen}, keywordstyle=\color{magenta},
    numberstyle=\tiny\color{codegray}, stringstyle=\color{codepurple}, basicstyle=\footnotesize,
    breakatwhitespace=false, breaklines=true, captionpos=b, keepspaces=true, numbers=left,                    
    numbersep=5pt, showspaces=false, showstringspaces=false, showtabs=false,tabsize=2
}
\lstset{style=mystyle}

\title{Tarea 06}

\begin{document}

\maketitle

\textbf{Primera parte (50 puntos)}

Guarde el siguiente código en un archivo llamado \texttt{grafica\_gauss.py}.

\begin{lstlisting}[language=Python, caption=grafica\_gauss.py]
import matplotlib.pyplot as plt
import math

n_sigma = 10
sigma = n_sigma * 0.05
n_points = 200
deltax = 20/n_points

l = list(range(n_points))
x = []
for i in l:
    x.append(-10.0 + deltax * i)

y = []
for i in x:
    y.append( math.exp(-0.5*i**2/sigma**2))
    
plt.figure()
plt.plot(x,y)    
plt.savefig('{}.png'.format(n_sigma))
plt.close()

\end{lstlisting}

Este código genera \verb"n_points" puntos entre $-10$ y $10$ para graficar una gaussiana
centrada en $0$ y $\sigma=$\verb"n_sigma"$\times 0.05$.
La gr\'afica se guarda como \verb"n_sigma.png".

\begin{itemize}
    \item Corra el c\'odigo una vez para comprobar su funcionamiento.
    \item Cambie \verb"n_sigma" (entero) para comprobar su funcionamiento.
    \item Modifique el código para que sea una funci\'on de nombre \verb"genera()"
      que tenga como argumento  de entrada \verb"n_sigma". 
      Con esto, al llamar la \verb"genera(n_sigma)" se genera la gráfica y la guarda.
    \item Guarde este archivo.
\end{itemize}

Cree un Notebook llamado \verb"ejercicio06.ipynb".

\begin{itemize}
    \item Importe el módulo \verb"grafica\_gauss" que creó en el numeral anterior.
    \item Llame la función que creó en el numeral anterior para comprobar que funciona.
    \item Inicie un contador $n$ que cambia entre $1$ y $50$ para llamar a la funci\'on
      \verb"genera(n)".
\end{itemize}
 
Ejecute la celda. Esto debería generar cerca de $49$ archivos de imágen. 


Ahora vamos a hacer una \verb"pelicula.gif" con las imágenes que se
crearon en el numeral anterior. 

\begin{itemize}
\item En una celda del notebook cree una lista con los nombres de las
  imágenes como strings. 
Llámela  \verb"filenames".
\item Use una estructura de código similar a la siguiente para que el
  módulo \verb"imageio" genere un nuevo archivo apilando las imágenes.
\end{itemize}



\begin{lstlisting}[language=Python, caption=gif animado]
import imageio

filenames = []  ## Lista con los nombres de los archivos

with imageio.get_writer("pelicula.gif", mode="I") as writer:
    for filename in filenames:
        image = imageio.imread(filename)
        writer.append_data(image)
\end{lstlisting}


\textbf{Segunda parte (50 puntos)}

En un archivo llamado \verb"grafica_arquimedes.py" escriba el c\'odigo 
necesario para tener una funci\'on \verb"espiral(n)" que grafica espiral de Ar\'quimedes como la mostrada 
en Wikipedia \url{https://en.wikipedia.org/wiki/Archimedean_spiral} donde el par\'ametro
$b=n/100.0$ y $a=0$.
Las funciones seno y coseno dentro de python las puede utilizar como \verb"math.sin()" y \verb"math.cos()"
luego de importar \verb"math".
La gr\'afica debe quedar guardada en un archivo \verb"png".

Dentro del notebook \verb"ejercicio06.ipynb".

\begin{itemize}
    \item Importe el módulo \verb"grafica\_arquimedes" que creó en el numeral anterior.
    \item Llame la función que creó en el numeral anterior para comprobar que funciona.
    \item Inicie un contador $n$ que cambia entre $1$ y $100$ para llamar a la funci\'on
      \verb"espiral(n)".
\end{itemize}
 
Ejecute la celda. Esto debería generar cerca de $100$ archivos de imágen. 
Con estos archivos haga un gif animado \verb"espiral.gif".


\end{document}
